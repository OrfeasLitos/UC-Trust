\section{Desire Satisfaction Ideal Functionality}
  Following the UC paradigm, in this section we define the ideal functionality for desire satisfaction, $\mathcal{F}_{SAT}$.
  In this setting, all the desires that are generated by the environment and are input to the players are immediately forwarded
  to $\mathcal{F}_{SAT}$; the functionality decides which desires to satisfy. Since the players are dummy and all desires are
  satisfied by the functionality, no trust semantics amongst the players are necessary.
  
  Nevertheless, given that all desires have a minimum cost, the cost semantics are still necessary, as we show with the
  following example: Consider a set of desires $D$ with more elements than the total number of input tokens all players have.
  $D$ could never be satisfied by the players because of the high total cost, but a $\mathcal{F}_{SAT}$ with no consideration
  for cost could in principle satisfy all desires in $D$.

  The functionality can calculate the properties and functions defined in~\ref{playerofdesire},~\ref{issatisfied}
  and~\ref{desiresofplayer}\ for all inputs at any moment in time.

  Without knowledge of the utilities the environment is going to give to each satisfied desire, the functionality may fail
  spectacularly. So knowledge of the utility of each desire, or at least some function of the utility given the desires is
  needed. We can assume that $\mathcal{F}_{SAT}$ knows $U$ or an approximation of it.
