\section{Desire Satisfaction Ideal Functionality}
  Following the UC paradigm, in this section we define the ideal functionality for desire satisfaction, $\mathcal{F}_{Desire}$.
  In this setting, all the desires that are generated by the environment and are input to the players are immediately forwarded
  to $\mathcal{F}_{Desire}$; the functionality decides which of them to satisfy. Since the players are dummy and all desires
  are satisfied by the functionality, no trust semantics are necessary. Nevertheless, given that all desires have a positive
  cost (at least one computational step), the cost semantics are still necessary, as we show with the following example:
  Consider a set of desires $D$ with more elements than the total number of input tokens all players have. $D$ could never be
  satisfied by the players because of the high total cost, but a $\mathcal{F}_{Desire}$ with no consideration for cost could in
  principle satisfy all desires in $D$. 
