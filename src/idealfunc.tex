\section{Desire Satisfaction Ideal Functionality}
  Following the UC paradigm, in this section we define the ideal functionality for desire satisfaction, $\mathcal{F}_{SAT}$.
  In this setting, all the desires that are generated by the environment and are input to the players are immediately forwarded
  to $\mathcal{F}_{SAT}$; the functionality decides which desires to satisfy. Since the players are dummy and all desires are
  satisfied by the functionality, no trust semantics amongst the players are necessary.
  
  Nevertheless, given that all desires have a minimum cost, the cost semantics are still necessary, as we show with the
  following example: Consider a set of desires $D$ with more elements than the total number of input tokens all players have.
  $D$ could never be satisfied by the players because of the high total cost, but a $\mathcal{F}_{SAT}$ with no consideration
  for cost could in principle satisfy all desires in $D$.

  The functionality can calculate the properties and functions defined in~\ref{playerofdesire},~\ref{issatisfied}
  and~\ref{desiresofplayer}\ for all inputs at any moment in time.

  Without knowledge of the utilities the environment is going to give to each satisfied desire, the functionality may fail
  spectacularly. So knowledge of the utility of each desire, or at least some function of the utility given the desires is
  needed. We can assume that $\mathcal{F}_{SAT}$ knows $U$ or an approximation of it.

  Going into more detail, $\mathcal{F}_{SAT}$ is a stateful process that acts as a market and as a bank for the players. The
  market does not offer a particular product for the same price to all users; For some users it may be cheaper than for others,
  reflecting the fact that some players can realize some desires more efficiently than others.

  $\mathcal{F}_{SAT}$ stores a number for each player that represents the amount of tokens this player has and a table with the
  price of each desire for each player. It provides the functions $cost\left(u, d\right)$ which returns the cost of the desire
  $d$ for player $u$ with no side effects, $sat\left(u, d\right)$ that returns the string that satisfies the desire $d$ to $u$
  and reduces the amount of the tokens belonging to $u$ by $cost\left(u, d\right)$. There exists also the function
  $transfer\left(u_1, u_2, t\right)$ which reduces the amount of tokens $u_1$ has by $t$ and increases the tokens of $u_2$ by
  $t$, given that initially the tokens belonging to $u_1$ were equal or more than $t$. This function is private to the
  functionality, thus can be used only internally.

  \subimport{algorithms/}{satfunc.tex}
