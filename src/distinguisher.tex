\section{$\mathcal{F}_{\mathrm{SAT}}$ and $\Pi_{\mathrm{SAT}}$ are potentially
distinguishable}
  Consider the hybrid world of Fig.~\ref{fig:allfuncs} (right) with $n$ ITMs executing
  $\Pi_{\mathrm{SAT}}$, where $\mathcal{F}_{\mathrm{Trust}}$ is replaced by
  $\mathcal{F}'_{\mathrm{Trust}}$:
  \subimport{algorithms/}{badtrustfunc.tex}
  We will show here that $\mathcal{E}$ can distinguish between
  $\mathcal{F}_{\mathrm{SAT}}$ and $\Pi_{\mathrm{SAT}}^{\mathcal{F}'_{\mathrm{Trust}}}$.
  \begin{proof}[Distinguishability]
    Consider the following adversary and environment:
    \subimport{algorithms/}{distadv.tex}
    \subimport{algorithms/}{distenv.tex}

    Because of the way $\mathcal{E}$ is built, there always exists a seller ($Bob$,
    line~\ref{distenv:Bob}) who has an asset (line~\ref{distenv:obtain:Bob}) that can
    satisfy the desire (line~\ref{distenv:satisfy}) of the buyer ($Alice$,
    line~\ref{distenv:Alice}).

    In case $\mathcal{E}$ interacts with $\mathcal{F}_{\mathrm{SAT}}$, let $\mathcal{S}$
    simulator that tries to simulate $\mathcal{A}$. $\mathcal{F}_{\mathrm{SAT}}$ will
    always send the message \texttt{(chooseBestSeller,} $\left\{\left(Bob, 1,
    s\right)\right\}$\texttt{,} $\left\{s\right\}$\texttt{,} $Alice$\texttt{)} to
    $\mathcal{S}$ because:
    \begin{enumerate}
      \item $Alice$ has one coin according to $\mathcal{E}$,
      line~\ref{distenv:obtain:Alice}, as required by $\mathcal{F}_{\textrm{SAT}}$,
      line~\ref{satfunc:hascoins}.
      \item It is in $Alice$'s benefit for the trade to go through, since she values
      acquiring one asset more than one coin
      (\texttt{util(}$Alice$\texttt{)(}$\left\{s\right\}, 0$\texttt{)} $= 2 > 1 = $
      \texttt{util(}$Alice$\texttt{)(}$\emptyset, 1$\texttt{)} as can be seen in
      $\mathcal{E}$, lines~\ref{distenv:setutil:Alice} and~\ref{distenv:sendutil}), as
      required in $\mathcal{F}_{\textrm{SAT}}$, line~\ref{satfunc:gains:Alice}.
      \item It is in $Bob$'s benefit for the trade to go through, since he values
      acquiring one coin more than one asset
      (\texttt{util(}$Bob$\texttt{)(}$\left\{s\right\}, 0$\texttt{)} $= 1 < 2 = $
      \texttt{util(}$Bob$\texttt{)(}$\emptyset, 1$\texttt{)} as can be seen in
      $\mathcal{E}$, lines~\ref{distenv:setutil:Bob} and~\ref{distenv:sendutil}), as
      required in $\mathcal{F}_{\textrm{SAT}}$, line~\ref{satfunc:gains:Bob}.
    \end{enumerate}

    $\mathcal{S}$ should always match the buyer and the seller because of the way
    $\mathcal{A}$ is built. More precisely, $\mathcal{S}$ must always respond to
    \texttt{(chooseBestSeller,} $\left\{\left(Bob, 1, s\right)\right\}$\texttt{,}
    $\left\{s\right\}$\texttt{,} \texttt{\_)} with \texttt{(bestSeller,}
    $\left\{\left(Bob, 1, s\right)\right\}$\texttt{,} $Bob$\texttt{,} 1\texttt{,}
    $s$\texttt{)} in order to correctly simulate $\mathcal{A}$
    (lines~\ref{distadv:Bob}-\ref{distadv:return}).

    Furthermore, $\mathcal{F}_{\mathrm{SAT}}$ never cheats on a trade and always chooses a
    suitable seller, price and asset (given that there exists one, which is the case here
    as we saw earlier) (lines~\ref{satfunc:badresponse}-\ref{satfunc:send:satisfied}),
    thus the exchange will always complete correctly and $\mathcal{E}$ will receive
    \texttt{satisfied} as response. $\mathcal{E}$ will always correctly output 1 (which
    corresponds to the functionality, line~\ref{distenv:func}).

    On the other hand, in case $\mathcal{E}$ interacts with $\Pi_{\mathrm{SAT}}$, then we
    observe that $\mathcal{F}'_{\mathrm{Trust}}$ does not choose players depending on
    their reputation (line~\ref{badtrust:Bob}), thus in this particular setting the
    utility of the players does not depend on their reputation. Thus, if
    $\mathcal{F}'_{\mathrm{Trust}}$ does not respond with $\bot$, it is always in $Bob$'s
    interest to cheat, since keeping the asset is preferrable to giving it ($\mathcal{E}$,
    lines~\ref{distenv:setutil:Bob} and~\ref{distenv:sendutil}). Thus $Alice$'s response
    to $\mathcal{E}$ will always be \texttt{cheated}. If $\mathcal{F}'_{\mathrm{Trust}}$
    responds with $\bot$ (line~\ref{badtrust:Bob}), the trade will not go through
    ($\Pi_{\mathrm{SAT}}$, lines~\ref{satprot:badtrustresponse}-\ref{satprot:unsat}) and
    $\mathcal{E}$ will receive \texttt{unsatisfied} as a response. In all cases
    $\mathcal{E}$ will correctly output 0 (which corresponds to the protocol,
    line~\ref{distenv:prot}).
  \end{proof}
