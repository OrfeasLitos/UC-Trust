\section{Introduction}
  Consider the UC setting, with an environment $\mathcal{E}$, an adversary $\mathcal{A}$ and a set of ITIs that follow a
  given protocol $\Pi$.

  \subimport{./definitions/}{player.tex}

  Intuitively, players spontaneously feel different desires of varying intensities and seek to satisfy them, either on their
  own, consuming part of their input tokens in the process, or by delegating the process to other players and paying them for
  their help with part of their input tokens. The choice depends on the perceived difference in price. Each player plays
  rationally, always attempting to maximize her utility.
  
  More precisely, let $\mathcal{D}$ be a (finite) set containing all possible desires. At arbitrary moments during execution,
  $\mathcal{E}$ can provide input to any player $Alice \in \mathcal{P}$ in the form $\left(idx, d\right)$, where $idx \in
  \mathbb{N}, d \in \mathcal{D}, u \in \mathbb{R}^{+}$. $idx$ represents an index number that is unique for each input
  generated by $\mathcal{E}$ and $d$ represents the desire. $d$ is satisfied when $Alice$ learns the string $s\left(idx, d,
  Alice\right)$, either by directly calculating it or by receiving it as subroutine output from another player. Some of the
  players, given as input the tuple $\left(idx, d, Alice\right)$, can calculate $s\left(idx, d, Alice\right)$ more efficiently
  than $Alice$, which means that they need to consume less input tokens than $Alice$ for this calculation. $Alice$ can choose
  to delegate this calculation to a more efficient player $Bob$ and provide the necessary input tokens for his computation with
  a surplus to compensate $Bob$ for his effort. Both players are better off, because $Alice$ spent less tokens than she would
  if she had calculated $s\left(idx, d, Alice\right)$ herself, whilst $Bob$ obtained some tokens which can in turn be used to
  satisfy some of his future desires.

  \subimport{./definitions/}{costofdesire.tex}

  It is reasonable to assume that there exists an absolute minimum of tokens that must be spent for the satisfaction of a
  desire, no matter how efficient the calculating party is.

  \subimport{./definitions/}{minimumcostofdesire.tex}

  \noindent Note that $1 \leq c_{min}\left(idx, d, Alice\right) \leq \min\limits_{v \in \mathcal{P}}{c\left(idx, d, Alice,
  v\right)}$.

  The game begins with all players being created by $\mathcal{E}$, each allocated a random amount of input tokens. The game
  ends at a moment specified by the $\mathcal{E}$, which is unknown to the players. At that moment $\mathcal{E}$ assigns a
  utility to each player depending on which desires were satisfied throughout the game.
