\section{Interaction with $F_{\mathrm{Trust}}$}
$Alice$ receives an offer from $Bob$ for price $p$. If $Trust_{Alice \rightarrow Bob} \geq
p$, then she pays Bob by telling $F_{\mathrm{Trust}}$ to redistribute the $p$ of her
direct trust to him so that we end up with $Trust_{Alice \rightarrow Bob} =
DirectTrust_{Alice \rightarrow Bob}$.

If the trade completes correctly and $Bob$ takes $p'$ coins from $Alice$'s direct trust,
where $0 \leq p' \leq p$, $Alice$ undoes the initial redistribution and on top of this she
adds another $p - p'$ as direct trust to $Bob$. If she has spare exclusive coins, she uses
them, otherwise she decreases her direct trust to the least recently used direct trust.
LRU direct trust is the oldest direct trust that helped her make a decision --- for that
she has to locally timestamp the direct trusts that are leveraged whenever she decides to
trust someone.

If the trade fails to complete and $Bob$ takes $p'$ coins from $Alice$'s direct trust,
where $0 \leq p' \leq p$, she takes her $p - p'$ coins back and also reduces her trust to
the players towards whom she had direct trusts that pointed her to $Bob$ by another total
$p$, keeping this money for exclusive use.

How can $Bob$ exploit this strategy? He would need the collaboration of the players
$Alice$ trusts directly. They have to steal from her at the same time as he does. If we
assume adaptive corruptions, then this is rather easy.

$Bob$ seems to be able to cheat on half of the trades and still he will have some incoming
direct trust. If he has say $100p$ incoming direct trust, then he can maintain the same
balance by selling honestly one item of value $p$ and then cheating on the next trade of
the same item alternatively. It is not directly obvious that this strategy will eventually
leave $Bob$ alone --- indeed, he may build a lucrative mafia-like scheme this way; he just
has to find constantly new first-line cheaters.

In order to mitigate this attack, a "know your customer" scheme may be necessary.  Indeed,
if honest sellers only sell to $Bob$ only if they trust him enough, this scheme does not
work because nobody will accept his money. Exclusive coins will only be useful as reserve
to be directly trusted to others. This however completely overhauls the way money works
today...
