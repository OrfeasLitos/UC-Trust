\section{A case against honest and malicious parties}
  Satisfying Definition~\ref{def:cheatsec} in case there exist honest parties and an
  Adversary seems impossible. When an honest buyer pays a malicious seller, then the
  malicious seller can always cheat and the honest buyer will report it to $\mathcal{E}$.
  It seems improbable that we can create an $\mathcal{F}_{\mathrm{Trust}}$ that returns a
  malicious party with negligible probability, especially since
  $\mathcal{F}_{\mathrm{Trust}}$ cannot know from the outset which parties are corrupted
  and which are not.

  Similarly, if a malicious buyer trades with an honest seller, then the seller will go
  through with the trade but the buyer will be able to falsely report a cheat to
  $\mathcal{E}$. This as well cannot be avoided, since $\mathcal{E}$ does not know whether
  the player to which it sends a $\left(\mathtt{satisfy, d, L}\right)$ message is
  corrupted.

  Nevertheless, if we completely change perspective and have all players be rational, then
  it is possible that for certain utility functions no player will cheat or report false
  cheating for fear of future punishment.
